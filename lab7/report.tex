\documentclass[12pt,a4paper]{article}

\usepackage[T1,T2A]{fontenc}
\usepackage[utf8]{inputenc}
\usepackage[english, russian]{babel}
\usepackage{indentfirst}
\usepackage{misccorr}
\usepackage{graphicx}
\usepackage{amsmath}
\usepackage{graphicx}
\usepackage{float}
\usepackage[left=20mm,right=10mm, top=20mm,bottom=20mm,bindingoffset=0mm]{geometry}

\setlength{\parskip}{6pt}
\DeclareGraphicsExtensions{.png}

\begin{document}

    \begin{titlepage}
        \begin{center}
            \large
            Санкт-Петербургский политехнический университет\\Петра Великого\\
            \vspace{0.5cm}
            Институт прикладной математики и механики\\
            \vspace{0.25cm}
            Кафедра «Прикладная математика»
            \vfill
            \textsc{\LARGE\textbf{Отчет по лабораторной работе №7}}\\[5mm]
            \Large
            по дисциплине\\"Математическая статистика"
        \end{center}
        \vfill
        \begin{tabular}{l p{175} l}
            Выполнила студентка\\группы 3630102/80201 && Деркаченко Анна Олеговна
            \vspace{0.25cm}
            \\Проверил\\доцент, к.ф.-м.н. && Баженов Александр Николаевич
        \end{tabular}
        \vfill
        \begin{center}
            Санкт-Петербург\\2021 г.
        \end{center}
    \end{titlepage}

\newpage
\begin{center}
    \tableofcontents
    \setcounter{page}{2}
\end{center}
\newpage
\begin{center}
    \listoffigures
\end{center}

\newpage
\section{Постановка задачи}
Дано нормальное распределение $N(x,0,1)$.

Необходимо:
\begin{enumerate}
    \item Сгенерировать выборку объемом 100 элементов
    \item Оценить параметры $\mu$ и $\sigma$ методом максимального правдоподобия, используя основную гипотезу $H_0$, что сгенерированное распределение имеет вид $N(x,\hat\mu,\hat\sigma)$
    \item Проверить основную гипотезу, используя критерий согласия $\chi^2$ и уровень значимости $\alpha=0.05$
    \item Привести таблицу вычислений $\chi^2$
    \item Исследовать точность (чувствительность) критерия $\chi^2$: сгенерировать выборки равномерного распределения и распределения Лапласа малого объема (например, 20 элементов) и проверить их на нормальность
\end{enumerate}

\section{Теория}
\subsection{Метод максимального правдоподобия}
Пусть $x_1,...,x_n$ — случайная выборка из генеральной совокупности с плотностью вероятности $f(x,\theta), L(x_1,...,x_n,\theta)$ — функция правдоподобия, представляющая собой совместную плотность вероятности независимых с.в. $x_1,...,x_n$ и рассматриваемая как функция неизвестного параметра $\theta$:
\begin{equation}
    L(x_1,...,x_n,\theta)=f(x_1,\theta)f(x_2,\theta)...f(x_n,\theta)
\end{equation}

\textit{Оценка максимального правдоподобия} - значение $\hat{\theta_{мп}}$ из множества допустимых значений параметра $\theta$, для которого функция правдоподобия принимает наибольшее значение при заданных $x_1,...,x_n$:
\begin{equation}
    \hat{\theta_{мп}}=\arg \max_\theta L(x_1,...,x_n,\theta)
\end{equation}

Часто проще искать максимум логарифма функции правдоподобия, так как он имеет максимум в одной точке с ней:
\begin{equation}
    \frac{\partial\ln{L}}{\partial\theta}=\frac{1}{L}\frac{\partial{L}}{\partial\theta},L>0
\end{equation}
и решать \textit{уравнение правдоподобия}
\begin{equation}
    \frac{\partial\ln{L}}{\partial\theta}=0
\end{equation}

\subsection{Проверка гипотезы о законе распределения генеральной совокупности. Метод хи-квадрат}
Пусть выдвинута гипотеза $H_0$ о виде закона распределения $F(x)$. Необходимо оценить его параметры и проверить закон в целом.

Для проверки гипотезы о законе распределения чаще всего применяется критерий согласия $\chi^2$. Пусть гипотетическая функция распределения $F(x)$ не содержит неизвестных параметров.

Разобьём генеральную совокупность, т.е. множество значений изучаемой случайной величины $X$ на $k$ непересекающихся равных подмножеств $\Delta_1,\Delta_2,...,\Delta_k$, где $k$ выбирается согласованным с $n$ и берется аналогичному при построении гистаграмм $k\approx1.72\sqrt[3]{n}$ или по формуле Старджесса $k\approx1+3.3\lg{n}$.

Пусть $p_i=P(X\in\Delta_i),i=\overline{1,k}$. Если генеральная совокупность - вся вещественная ось, $p_i=F(a_i)-F(a_{i-1}),i=\overline{1,k}$. При этом $\sum_{i=1}^k{p_i}=1$ и $p_i>0,i=\overline{1,k}$.

Пусть, далее, $n_1,n_2,..,n_k$ — частоты попадания выборочных элементов в подмножества $\Delta_1,\Delta_2,...,\Delta_k$ соответственно.

Если гипотеза $H_0$ справедлива, то относительные частоты $\frac{n_i}{n}\rightarrow p_i, i=\overline{1,k}$. Следовательно, мера отклонения выборочного распределения от гипотетического с использованием коэффициентов Пирсона:
\begin{equation}
    \chi^2=\sum_{i=1}^k{\frac{{n_i-np_i}^2}{np_i}}
\end{equation}

\textit{Теорема К.Пирсона:} статистика критерия $\chi^2$ асимптотически распределена по закону $\chi^2$ с $k-1$ степенями свободы.

\textbf{Правило проверки гипотезы о законе распределения по методу $\chi^2$}
\begin{enumerate}
    \item Выбираем уровень значимости $\alpha$
    \item Находим квантиль $\chi^2_{1-\alpha}(k - 1)$ распределения хи-квадрат с $k-1$ степенями свободы порядка $1-\alpha$
    \item С помощью гипотетической функции распределения $F(x)$ вычисляем вероятности $p_i=P(X\in\Delta_i),i=\overline{1,k}$
    \item Находим частоты $n_i$ попадания элементов выборки в подмножества $\Delta_i,i=\overline{1,k}$
    \item Вычисляем выборочное значение статистики критерия $\chi^2$
    \item Сравниваем $\chi^2_B$ и квантиль $\chi^2_{1-\alpha}(k-1):$
        \begin{itemize}
            \item если $\chi^2_B<\chi^2_{1-\alpha}(k-1)$, то гипотеза $H_0$ на данном этапе проверки принимается
            \item иначе гипотеза $H_0$ отвергается, выбирается одно из альтернативных распределений, и процедура проверки повторяется
        \end{itemize}
\end{enumerate}

\textit{Замечание:} при ситуации $\chi^2_B\approx\chi^2_{1-\alpha}(k-1)$ стоит увеличить объем выборки (например, в 2 раза), чтобы требуемое неравенство было более четким.

\textit{Замечание:} Изучено, что если для каких-либо подмножеств $\Delta_i,i=\overline{1,k}$ условие $np_i\geq5$ не выполняется, то следует объединить соседние подмножества (промежутки). Это условие выдвигается требованием близости величин $\frac{(n_i-np_i)}{\sqrt{np_i}}$. Тогда случайная величина будет распределена по закону, близкому к хи-квадрат. Такая близость обеспечивается достаточной численностью элементов в подмножествах $\Delta_i$.


\section{Реализация}
Реализация лабораторной работы проводилась на языке Python в среде разработки PyCharm c использованием дополнительных библиотек:
\begin{itemize}
    \item scipy
    \item numpy
    \item matplotlib
    \item math
\end{itemize}

Исходный код лабораторной работы размещен в GitHub-репозитории.

URL: https://github.com/derkanw/Mathstat/tree/main/lab7

\section {Результаты}
\subsection{Выборка нормального распределения}
\begin{equation}
    \left\{
    \begin{array}{ll}
        \hat{\mu}=0.0938\\
        \hat{\sigma}=0.962\\
        \text{Количество промежутков }k=8\\
        \text{Уровень значимости }\alpha=0.05\\
    \end{array}
    \right.
\end{equation}

Тогда квантиль $\chi^2_{1-\alpha}(k-1)=\chi^2_{0.95}\approx14.0671$. 
\begin{table}[H]
    \centering
    \begin{tabular}{|c|c|c|c|c|c|c|}
    \hline
    $i$ & Границы $\Delta_i$ & $n_i$ & $p_i$ & $np_i$ & $n_i-np_i$ & $\frac{(n_i-np_i)^2}{np_i}$\\\hline\hline
    1 & [$-\infty$, -1.1] & 10 & 0.1357 & 13.5666 & -3.5666 & 0.9376\\\hline
    2 & [-1.1, -0.7333] & 7 & 0.096 & 9.6012 & -2.6012 & 0.7047\\\hline
    3 & [-0.7333, -0.3667] & 14 & 0.1253 & 12.5256 & 1.4744 & 0.1735\\\hline
    4 & [-0.3667, 0.0] & 20 & 0.1431 & 14.3066 & 5.6934 & 2.2657\\\hline
    5 & [0.0, 0.3667] & 12 & 0.1431 & 14.3066 & -2.3066 & 0.3719\\\hline
    6 & [0.3667, 0.7333] & 14 & 0.1253 & 12.5256 & 1.4744 & 0.1735\\\hline
    7 & [0.7333, 1.1] & 9 & 0.096 & 9.6012 & -0.6012 & 0.0376\\\hline
    8 & [1.1, $\infty$] & 14 & 0.1357 & 13.5666 & 0.4334 & 0.0138\\\hline
    $\sum$ & - & 100 & 1 & 100 & 0 & 4.6785\\\hline
    \end{tabular}
    \caption{Вычисление $\chi^2_B$ при нормальном распределении}
\end{table}

\subsection{Выборка распределения Лапласа}
\begin{equation}
    \left\{
    \begin{array}{ll}
        \hat{\mu}=-0.1291\\
        \hat{\sigma}=0.8581\\
        \text{Количество промежутков }k=5\\
        \text{Уровень значимости }\alpha=0.05\\
    \end{array}
    \right.
\end{equation}

Тогда квантиль $\chi^2_{1-\alpha}(k-1)=\chi^2_{0.95}\approx9.4877$. 
\begin{table}[H]
    \centering
    \begin{tabular}{|c|c|c|c|c|c|c|}
    \hline
    $i$ & Границы $\Delta_i$ & $n_i$ & $p_i$ & $np_i$ & $n_i-np_i$ & $\frac{(n_i-np_i)^2}{np_i}$\\\hline\hline
    1 & [$-\infty$, -1.1] & 2 & 0.1357 & 2.7133 & -0.7133 & 0.1875\\\hline
    2 & [-1.1, -0.3667] & 5 & 0.2213 & 4.4254 & 0.5746 & 0.0746\\\hline
    3 & [-0.3667, 0.3667] & 9 & 0.2861 & 5.7226 & 3.2774 & 1.8769\\\hline
    4 & [0.3667, 1.1] & 1 & 0.2213 & 4.4254 & -3.4254 & 2.6513\\\hline
    5 & [1.1, $\infty$] & 3 & 0.1357 & 2.7133 & 0.2867 & 0.0303\\\hline
    $\sum$ & - & 20 & 1 & 20 & 0 & 4.8207\\\hline
    \end{tabular}
    \caption{Вычисление $\chi^2_B$ при нормальном распределении}
\end{table}

\subsection{Выборка равномерного распределения}
\begin{equation}
    \left\{
    \begin{array}{ll}
        \hat{\mu}=-0.2113\\
        \hat{\sigma}=0.885\\
        \text{Количество промежутков }k=5\\
        \text{Уровень значимости }\alpha=0.05\\
    \end{array}
    \right.
\end{equation}

Тогда квантиль $\chi^2_{1-\alpha}(k-1)=\chi^2_{0.95}\approx9.4877$. 
\begin{table}[H]
    \centering
    \begin{tabular}{|c|c|c|c|c|c|c|}
    \hline
    $i$ & Границы $\Delta_i$ & $n_i$ & $p_i$ & $np_i$ & $n_i-np_i$ & $\frac{(n_i-np_i)^2}{np_i}$\\\hline\hline
    1 & [$-\infty$, -1.1] & 3 & 0.1357 & 2.7133 & 0.2867 & 0.0303\\\hline
    2 & [-1.1, -0.3667] & 7 & 0.2213 & 4.4254 & 2.5746 & 1.4979\\\hline
    3 & [-0.3667, 0.3667] & 4 & 0.2861 & 5.7226 & -1.7226 & 0.5186\\\hline
    4 & [0.3667, 1.1] & 4 & 0.2213 & 4.4254 & -0.4254 & 0.0409\\\hline
    5 & [1.1, $\infty$] & 2 & 0.1357 & 2.7133 & -0.7133 & 0.1875\\\hline
    $\sum$ & - & 20 & 1 & 20 & 0 & 2.2752\\\hline
    \end{tabular}
    \caption{Вычисление $\chi^2_B$ при нормальном распределении}
\end{table}

\section{Обсуждение}
Для всех случаев справедливо неравенство $\chi^2_B<\chi^2_{0.95}$. Для случая выборки, распределенной по закону $N(x,\hat{\mu},\hat{\sigma})$, можно сказать, что гипотеза $H_0$ о нормальном распределении на уровне значимости $\alpha=0.05$ согласуется с ней с определенной точностью. Для случаев распределения Лапласа и равномерного распределения эта оценка немного хуже приближается к параметрам нормального распределения, что объясняется малым количеством выборки. В итоге для всех случаев гипотеза $H_0$ принята.
\end{document}
