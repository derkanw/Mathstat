\documentclass[12pt,a4paper]{article}

\usepackage[T1,T2A]{fontenc}
\usepackage[utf8]{inputenc}
\usepackage[english, russian]{babel}
\usepackage{indentfirst}
\usepackage{misccorr}
\usepackage{graphicx}
\usepackage{amsmath}
\usepackage{graphicx}
\usepackage{float}
\usepackage[left=20mm,right=10mm, top=20mm,bottom=20mm,bindingoffset=0mm]{geometry}

\setlength{\parskip}{6pt}
\DeclareGraphicsExtensions{.png}

\begin{document}

    \begin{titlepage}
        \begin{center}
            \large
            Санкт-Петербургский политехнический университет\\Петра Великого\\
            \vspace{0.5cm}
            Институт прикладной математики и механики\\
            \vspace{0.25cm}
            Кафедра «Прикладная математика»
            \vfill
            \textsc{\LARGE\textbf{Отчет по лабораторной работе №2}}\\[5mm]
            \Large
            по дисциплине\\"Математическая статистика"
        \end{center}
        \vfill
        \begin{tabular}{l p{175} l}
            Выполнила студентка\\группы 3630102/80201 && Деркаченко Анна Олеговна
            \vspace{0.25cm}
            \\Проверил\\доцент, к.ф.-м.н. && Баженов Александр Николаевич
        \end{tabular}
        \vfill
        \begin{center}
            Санкт-Петербург\\2021 г.
        \end{center}
    \end{titlepage}

\newpage
\begin{center}
    \tableofcontents
    \setcounter{page}{2}
\end{center}
\newpage
\begin{center}
    \listoffigures
\end{center}

\newpage
\section{Постановка задачи}
Даны распределения:
\begin{itemize}
    \item нормальное распределение $N(x,0,1)$
    \item распределение Коши $C(x,0,1)$
    \item распределение Лапласа $L(x,0,\frac{1}{\sqrt{2}})$
    \item распределение Пуассона $P(k,10)$
    \item равномерное распределение $U(x,-\sqrt{3},\sqrt{3})$
\end{itemize}

Необходимо:
\begin{enumerate}
    \item Сгенерировать выборки размером 10, 100 и 1000 элементов
    \item Построить для них характеристики положения данных: $\overline{x}, med x, z_R, z_Q, z_tr$
    \item Повторить данные вычисления 1000 раз для каждой выборки, найти среднее характеристик положения $E(z)=\overline{z}$ и вычислить оценку дисперсии $D(z)=\overline{z^2}-{\overline{z}}^2$
    \item Представить полученные результаты в виде таблиц
\end{enumerate}

\section{Теория}
\subsection{Распределения}
Плотности рассматриваемых распределений:
\begin{itemize}
		\item нормальное распределение
		    \begin{equation}
			    N(x,0,1)=\frac{1}{\sqrt{2\pi}}e^{-\frac{x^2}{2}}
			    \label{normal} 
			\end{equation}
		\item распределение Коши
		    \begin{equation}
				C(x,0,1)=\frac{1}{\pi}\frac{1}{x^2+1}
				\label{cauchy}
			\end{equation} 
		\item распределение Лапласа
		    \begin{equation}
				L(x,0,\frac{1}{\sqrt{2}})=\frac{1}{\sqrt{2}}e^{-\sqrt{2}|x|}
				\label{laplace} 
			\end{equation}
		\item распределение Пуассона
		    \begin{equation}
				P(k,10)=\frac{10^k}{k!}e^{-10}
				\label{poisson}
			\end{equation}
		\item равномерное распределение
		    \begin{equation}
				U(x,-\sqrt{3},\sqrt{3})=
				\begin{cases}
					\frac{1}{2\sqrt{3}},|x|\leq\sqrt{3}\\0,|x|>\sqrt{3}
				\end{cases}
				\label{uniform}
			\end{equation}
\end{itemize}

\subsection{Выборочные числовые характеристики}
\textit{Вариационный ряд} - последовательность элементов выборки $x_1,x_2,...,x_n$, расположенных в неубывающем порядке.

Дискретная случайная величина имеет числовые характеристики, образующиеся с помощью выборки из этой величины.
\subsubsection{Характеристики положения}
\begin{itemize}
    \item Выборочное среднее
        \begin{equation}
            \overline{x}=\frac{1}{n}\sum_{i=1}^{n}{x_i}
		\end{equation}
	\item Выборочная медиана
	    \begin{equation}
			med x=
			\begin{cases}
			    x_{l+1},n=2l+1\\
				\frac{x_l+x_{l+1}}{2},n=2l
			\end{cases}
		\end{equation}
	\item Полусумма экстремальных выборочных элементов
	    \begin{equation}
			z_R=\frac{x_1 + x_n}{2}
		\end{equation}
	\item Полусумма квартилей
	    \newline Выборочная квартиль $z_p$ порядка $p$ определяется формулой
	    \begin{equation}
		    z_p =
			\begin{cases}
			    x_{[np]+1},np-\text{дробное}\\
		      	x_{np},np-\text{целое}
	        \end{cases}
		\end{equation}
	    Полусумма квартилей
	    \begin{equation}
			z_Q=\frac{z_{1/4}+z_{3/4}}{2}
		\end{equation}
	\item Усечённое среднее
	    \begin{equation}
			z_{tr}=\frac{1}{n-2r}\sum_{i=r+1}^{n-r}{x_i}, r\approx\frac{n}{4}
		\end{equation}
\end{itemize}

\subsubsection{Характеристики рассеивания}
Выборочная дисперсия определяется по формуле:
\begin{equation}
    D=\frac{1}{n}\sum^{n}_{i=1}{(x_i-\overline{x})^2}
\end{equation}

\section{Реализация}
Реализация лабораторной работы проводилась на языке Python в среде разработки PyCharm c использованием дополнительных библиотек:
\begin{itemize}
    \item scipy
    \item numpy
    \item math
\end{itemize}

Исходный код лабораторной работы размещен в GitHub-репозитории.

URL: https://github.com/derkanw/Mathstat/tree/main/lab2

\section {Результаты}
\begin{table}[H]
    \centering
    \begin{tabular}{|l||c|c|c|c|c|}
        \hline
        & $\overline{x}$ & $med x$ & $z_R$ & $z_Q$ & $z_{tr}$\\\hline\hline
        n=10 & & & & &\\\hline
        $E(z)$ & 0.006371 & -0.007288 & 0.023028 & 0.321795 & 0.279379\\\hline
        $D(z)$ & 0.099103 & 0.148119 & 0.167815 & 0.121352 & 0.121102\\\hline
        n=100 & & & & &\\\hline
        $E(z)$ & -0.00237 & -0.004485 & -0.013181 & 0.011586 & 0.023726\\\hline
        $D(z)$ & 0.010205 & 0.014457 & 0.081592 & 0.012794 & 0.011714\\\hline
        n=1000 & & & & &\\\hline
        $E(z)$ & -8.6e-05 & -0.000496 & -0.007813 & 0.001156 & 0.002173\\\hline
        $D(z)$ & 0.001049 & 0.001548 & 0.062136 & 0.001285 & 0.001217\\\hline
    \end{tabular}
    \caption{Таблица характеристик для нормального распределения}
    \label{tab:normal}
\end{table}

\begin{table}[H]
    \centering
    \begin{tabular}{|l||c|c|c|c|c|}
        \hline
        & $\overline{x}$ & $med x$ & $z_R$ & $z_Q$ & $z_{tr}$\\\hline\hline
        n=10 & & & & &\\\hline
        $E(z)$ & -4.381265 & -0.011736 & -21.876557 & 1.167523 & 0.709626\\\hline
        $D(z)$ & 27051.784196 & 0.340525 & 676317.178264 & 4.667338 & 1.252677\\\hline
        n=100 & & & & &\\\hline
        $E(z)$ & -1.156414 & -0.00829 & -54.318545 & 0.022735 & 0.034599\\\hline
        $D(z)$ & 2465.615699 & 0.025737 & 5999973.185907 & 0.056302 & 0.028603\\\hline
        n=1000 & & & & &\\\hline
        $E(z)$ & 0.624531 & 0.000828 & 316.572342 & 0.002332 & 0.004516\\\hline
        $D(z)$ & 391.702877 & 0.002398 & 95371869.306265 & 0.004996 & 0.002449\\\hline
    \end{tabular}
    \caption{Таблица характеристик для распределения Коши}
    \label{tab:cauchy}
\end{table}

\begin{table}[H]
    \centering
    \begin{tabular}{|l||c|c|c|c|c|}
        \hline
        & $\overline{x}$ & $med x$ & $z_R$ & $z_Q$ & $z_{tr}$\\\hline\hline
        n=10 & & & & &\\\hline
        $E(z)$ & -0.002224 & -0.002246 & -0.004543 & 0.296567 & 0.230068\\\hline
        $D(z)$ & 0.101624 & 0.075864 & 0.384596 & 0.118392 & 0.084\\\hline
        n=100 & & & & &\\\hline
        $E(z)$ & -0.002469 & -0.000214 & -0.036222 & 0.011354 &  0.019459\\\hline
        $D(z)$ & 0.01011 & 0.005916 & 0.427356 & 0.010273 & 0.006263\\\hline
        n=1000 & & & & &\\\hline
        $E(z)$ & -0.00023 & -0.000612 & 0.023942 & 0.001222 & 0.001819\\\hline
        $D(z)$ & 0.001008 & 0.000537 & 0.409581 & 0.000964 & 0.000605\\\hline
    \end{tabular}
    \caption{Таблица характеристик для распределения Лапласа}
    \label{tab:laplace}
\end{table}

\begin{table}[H]
    \centering
    \begin{tabular}{|l||c|c|c|c|c|}
        \hline
        & $\overline{x}$ & $med x$ & $z_R$ & $z_Q$ & $z_{tr}$\\\hline\hline
        n=10 & & & & &\\\hline
        $E(z)$ & 9.9858 & 9.857 & 10.2595 & 10.901 & 10.756667\\\hline
        $D(z)$ & 1.109498 & 1.527051 & 1.96991 & 1.479199 & 1.343289\\\hline
        n=100 & & & & &\\\hline
        $E(z)$ & 9.98466 & 9.8115 & 10.973 & 9.956 & 9.9234\\\hline
        $D(z)$ & 0.0984 & 0.216718 & 0.993271 & 0.154564 & 0.1187\\\hline
        n=1000 & & & & &\\\hline
        $E(z)$ & 10.000605 & 9.996 & 11.6505 & 9.9985 & 9.866522\\\hline
        $D(z)$ & 0.00996 & 0.003484 & 0.6741 & 0.001248 & 0.011262\\\hline
    \end{tabular}
    \caption{Таблица характеристик для распределения Пуассона}
    \label{tab:poisson}
\end{table}

\begin{table}[H]
    \centering
    \begin{tabular}{|l||c|c|c|c|c|}
        \hline
        & $\overline{x}$ & $med x$ & $z_R$ & $z_Q$ & $z_{tr}$\\\hline\hline
        n=10 & & & & &\\\hline
        $E(z)$ & 0.007634 & -0.003641 & 0.00133 & 0.31847 & 0.322529\\\hline
        $D(z)$ & 0.100261 & 0.231998 & 0.046272 & 0.124854 & 0.150706\\\hline
        n=100 & & & & &\\\hline
        $E(z)$ & -0.00029 & 0.000115 & -0.000439 & 0.014805 & 0.033771\\\hline
        $D(z)$ & 0.01091 & 0.029984 & 0.00057 & 0.016426 & 0.021133\\\hline
        n=1000 & & & & &\\\hline
        $E(z)$ & 0.001661 & 0.002659 & -0.000108 & 0.003872 & 0.005872\\\hline
        $D(z)$ & 0.001096 & 0.003206 & 6e-06 & 0.001635 & 0.00219\\\hline
    \end{tabular}
    \caption{Таблица характеристик для равномерного распределения}
    \label{tab:uniform}
\end{table}

\section{Обсуждение}
Исходя из данный, представленных в таблицах, можно сделать вывод, что увеличение размерности выборки имеет уточняющее значение для выборочной оценки характеристик случайной величины. Также сходные значения соответсвующих параметров имеют нормальное распределение, распределение Лапласа и равномерное распределение. Примечательно, что большая часть их значений близка к нулю.

Стоит отметить, что распределение Пуассона имеет значением среднего своих параметрах при 1000 опытах величину в окрестности 10, что подтверждается значением параметра задания данного распределения.

К тому же, в таблице характеристик распределения Коши можно выделить аномальные значение, явно превышающие ожидаемые. Такую ситуацию можно объяснить наличием различных выбросов, неопределенностью математического ожидания и бесконечностью дисперсии случайной величины, распределенной по данному закону.
\end{document}
